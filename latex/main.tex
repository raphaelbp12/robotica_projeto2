%%% Template originaly created by Karol Kozioł (mail@karol-koziol.net) and modified for ShareLaTeX use

\documentclass[a4paper,11pt]{article}

\usepackage{indentfirst}
\usepackage[T1]{fontenc}
\usepackage[utf8]{inputenc}
\usepackage{graphicx}
\graphicspath{ {./images/} }

\usepackage{xcolor}

\renewcommand\familydefault{\sfdefault}
\usepackage{tgheros}

\usepackage{amsmath,amssymb,amsthm,textcomp,latexsym}
\usepackage{enumerate}
\usepackage{multicol}
\usepackage{tikz}

\usepackage{caption}
\usepackage{subcaption}
\usepackage{epstopdf}
\usepackage{cancel}
\usepackage{mathtools}


\usepackage{geometry}
\geometry{total={210mm,297mm},
left=25mm,right=25mm,%
bindingoffset=0mm, top=20mm,bottom=20mm}


\linespread{1.3}

\newcommand{\linia}{\rule{\linewidth}{0.5pt}}

% custom theorems if needed
\newtheoremstyle{mytheor}
    {1ex}{1ex}{\normalfont}{0pt}{\scshape}{.}{1ex}
    {{\thmname{#1 }}{\thmnumber{#2}}{\thmnote{ (#3)}}}

\theoremstyle{mytheor}
\newtheorem{defi}{Definition}

% my own titles
\makeatletter
\renewcommand{\maketitle}{
\begin{center}
\vspace{2ex}
{\huge \textsc{\@title}}
\vspace{1ex}
\\
\linia\\
\@author \hfill \@date
\vspace{4ex}
\end{center}
}
\makeatother
%%%

% custom footers and headers
\usepackage{fancyhdr}
\pagestyle{fancy}
\lhead{Projeto 2}
\chead{}
\rhead{2019.2}
\lfoot{}
\cfoot{}
\rfoot{Page \thepage}
\renewcommand{\headrulewidth}{0pt}
\renewcommand{\footrulewidth}{0pt}
%

% code listing settings
\usepackage{listings}
\lstset{
    language=Matlab,
    basicstyle=\ttfamily\small,
    aboveskip={1.0\baselineskip},
    belowskip={1.0\baselineskip},
    columns=fixed,
    extendedchars=true,
    breaklines=true,
    tabsize=4,
    prebreak=\raisebox{0ex}[0ex][0ex]{\ensuremath{\hookleftarrow}},
    frame=lines,
    showtabs=false,
    showspaces=false,
    showstringspaces=false,
    keywordstyle=\color[rgb]{0.627,0.126,0.941},
    commentstyle=\color[rgb]{0.133,0.545,0.133},
    stringstyle=\color[rgb]{01,0,0},
    numbers=left,
    numberstyle=\small,
    stepnumber=1,
    numbersep=10pt,
    captionpos=t,
    escapeinside={\%*}{*)}
}

%%%----------%%%----------%%%----------%%%----------%%%


\DeclarePairedDelimiter\abs{\lvert}{\rvert}%
\DeclarePairedDelimiter\norm{\lVert}{\rVert}%

% Swap the definition of \abs* and \norm*, so that \abs
% and \norm resizes the size of the brackets, and the 
% starred version does not.
\makeatletter
\let\oldabs\abs
\def\abs{\@ifstar{\oldabs}{\oldabs*}}
%
\let\oldnorm\norm
\def\norm{\@ifstar{\oldnorm}{\oldnorm*}}
\makeatother

\begin{document}

\title{Robótica e Automação}

\author{Raphael Barros Parreira}

\date{}

\maketitle

\section{Controle Cinemático de Manipuladores} % 1

O exercício tem como objetivo o \textbf{controle cinemático} da posição do punho do manipulador, sem se preocupar com a orientação, nesse primeiro momento. Para isso, o manipulador escolhido foi um 7R, com a tabela de Denavit-Hartenberg (tabela \ref{tab:ex1_dh}) e a posição inicial ($q_0$ representada na figura \ref{fig:ex1_ready}) dadas.

Além disso, o enunciado define 3 trajetórias distintas, define os controles a serem implementados e insere restrições.

Como o objetivo é apenas controlar a posição do manipulador, pode-se ignorar as 3 últimas juntas, já que, pela tabela de DH, elas apenas influenciam na orientação do manipulador. Essa decisão implicará numa série de simplificações feitas no sistema, começando por sempre definir a posição e a velocidade dessas 3 juntas como $zero$.


\begin{figure}[!ht]
\centering
\includegraphics[width=0.5\textwidth]{figs/ex1_ready.pdf}
\caption{Manipulador Antropomórfico na posição inicial}
\label{fig:ex1_ready}
\end{figure}

\begin{gather*}
\theta_5 = \theta_6 = \theta_7 = 0 \\
\dot{\theta_5} = \dot{\theta_6} = \dot{\theta_7} = 0 \\
q_0 = [0, 0, 0, \pi/2] \\
\dot{\theta}_{max} = 3 rad/s \Rightarrow \norm{u_i} \leq 3 (i = 1,2,3,4)
\end{gather*}

\begin{table}[!ht]
\centering
\caption{Tabela de Denavit-Hartenberg do Manipulador Antropomórfico.}
\label{tab:ex1_dh}

\begin{tabular}{|c|c|c|c|c|c|}
\hline
Junta  & $\alpha (rad)$ & $A (m)$ & $\theta (rad)$ & $D (m)$ & $Offset (rad)$ \\ \hline
$1$  & $\pi/2$ & 0 & $\theta_1$ & 0 & 0 \\ \hline
$2$  & $\pi/2$ & 0 & $\theta_2$ & 0 & $\pi$ \\ \hline
$3$  & $\pi/2$ & 0 & $\theta_3$ & $-0.4208$ & $\pi$ \\ \hline
$4$  & $\pi/2$ & 0 & $\theta_4$ & 0 & $\pi$ \\ \hline
$5$  & $\pi/2$ & 0 & $\theta_5$ & $-0.3143$ & $\pi$ \\ \hline
$6$  & $\pi/2$ & 0 & $\theta_6$ & 0 & $\pi$ \\ \hline
$7$  & $\pi$   & 0 & $\theta_7$ & 0 & $\pi$ \\ \hline

\end{tabular}
\end{table}

\subsection{Controle}


No controle cinemático assume-se que a velocidade do manipulador é a variável manipulada e que a \textbf{dinâmica} do manipulador pode ser \textbf{desprezada}. Além disso, considera-se que a velocidade das juntas responde instantaneamente ao sinal de entrada do motor da mesma, implicando na condição de $ u \approx \dot{\theta} $

\begin{gather*}
\begin{bmatrix} \vec{v} \\ \vec{\omega} \end{bmatrix} = J(\theta)\dot{\theta} \quad \Rightarrow \quad  \begin{bmatrix} \vec{v} \\ \vec{\omega} \end{bmatrix} = J(\theta)u\\
u = \dot{\theta} \quad \Rightarrow \quad \dot{x} = J(\theta)u \\
x(t) \longrightarrow x_d(t) \\
e = x - x_d \\
\end{gather*}

O objetivo de controle é que o manipulador siga uma trajetória pré-determinada. Portanto, já que se conhece a trajetória, por conseguinte, também se conhece a sua derivada. O que permite adicionar informações no sistema de controle (ação \textit{feed-forward}, tornando possível a \textbf{liminação do erro em estado estacionário} ao seguir uma trajetória.

Para que os cálculos sejam feitos, é necessário escolher uma base para expressar os vetores e calcular o Jacobiano. Já que a trajetória dada está expressa na base do sistema inercial, uma boa escolha é usar essa mesma base.

Uma boa ideia para a lei de controle é eliminar as não-linearidades do sistema, incluindo-se a inversa do Jacobiano no sinal de controle.

\begin{gather*}
\dot{x} = J\bar{u} \\
\bar{u} = J^{-1} [ \dot{x_d} + K (x_d - x) ] \\
\dot{x} = JJ^{-1} [ \dot{x_d} + K (x_d - x) ] \\
\dot{x} - \dot{x_d} = K (x_d - x) \\
\dot{e}(t)= -Ke(t)  \\
e \longrightarrow 0, \quad t \longrightarrow \infty
\end{gather*}

Como no caso de um manipulador com 7 juntas, o Jacobiano não é uma matriz quadrada, ele não possui inversa. Sendo assim, se faz necessário o uso da \textbf{pseudo-inversa} ($ J^\dagger $).


\begin{gather*}
J^\dagger = J^T(JJ^T)^{-1} \quad \Rightarrow \quad JJ^\dagger = I \\
\bar{u} = J^{\dagger} [ \dot{x_d} + K (x_d - x) ] \\
\end{gather*}


\subsection{Trajetórias}

O enunciado considera 3 trajetórias de referência distintas, com $w_n = 2\pi/10 $. Além disso, não se pode usar derivadores puros. Logo, as derivadas das trajetórias foram calculadas como se segue:

\renewcommand{\labelenumi}{\alph{enumi}}
 \begin{enumerate}
   \item Trajetória a
   	\begin{gather*}
	x_{d} = 
	\begin{bmatrix} 
	0.075 (sin(w_n t) + sin( 4 w_n t)) + 0.456 \\ 
	0 \\ 
	0.075 (cos(w_n t) + cos( 4 w_n t)) + 0.069 
	\end{bmatrix} \\
	\dot{x_{d}} = 
	\begin{bmatrix}
	0.075 w_n (cos(w_n t) + 4 cos( 4 w_n t)) \\ 
	0 \\ 
	-0.075 w_n (sin(w_n t) + 4 sin( 4 w_n t)) 
	\end{bmatrix} \\
	\end{gather*}
   \item Trejetória b
   	\begin{gather*}
	x_{d} = 
	\begin{bmatrix} 
	0.075 sin(w_n t) + 0.456 \\ 
	0 \\ 
	0.075 cos(w_n t) + 0.2 
	\end{bmatrix} \\
	\dot{x_{d}} = 
	\begin{bmatrix} 
	0.075 w_n cos(w_n t) \\ 
	0 \\ 
	- 0.075 w_n sin(w_n t) 
	\end{bmatrix} \\
	\end{gather*}
   \item Trejetória c
   	\begin{gather*}
	x_{d} = 
	\begin{bmatrix} 
	0.075 sin(w_n t) + 0.456 \\ 
	0.075 (sin(w_n t) + cos(w_n t)) \\ 
	0.075 cos(w_n t) + 0.2 
	\end{bmatrix} \\
	\dot{x_{d}} = 
	\begin{bmatrix} 
	0.075 w_n cos(w_n t) \\ 
	0.075 w_n (cos(w_n t) - sin(w_n t)) \\ 
	- 0.075 w_n sin(w_n t) 
	\end{bmatrix} \\
	\end{gather*}
 \end{enumerate}


\subsection{Simulação}

O sistema representado na topologia da figura \ref{fig:ex1_simulink} está simulando a lei de controle com o feed-forward. Porém, para efeitos de comparação, o exercício pede para desligar essa ação. Para isso, basta apenas excluir o sinal de $xddot$ que é somado depois do ganho.

Cada uma das trajetórias foi simulada com e sem ação do feed-forward.

\begin{figure}[!ht]
\centering
\includegraphics[width=1\textwidth]{figs/ex1_simulink}
\caption{Topologia do Simulink.}
\label{fig:ex1_simulink}
\end{figure}


\subsection{Ganhos}

Um ponto importante do trabalho é o ganho $K$ do controlador. Isso se dá porque foi pedido o maior ganho possível atendendo à especificação de $ \norm{u_i} \leq 3 (i = 1,2,3,4)$.

Sendo assim, se faz necessário calcular a ordem de grandeza do ganho e depois se fazer o ajuste fino para cada um dos casos.

\begin{gather*}
\norm{\bar{u}} \leq 3 \quad \Rightarrow \quad \norm{J^{\dagger} [ \dot{x_d} + K (x_d - x) ]} \leq 3 \\
\norm{J^{\dagger}}_{max} [ \norm{\dot{x_d}}_{max} + K_{base} \norm{e}_{max} ] = 3\\
K_{base} = \dfrac{\dfrac{3}{\norm{J^\dagger}_{max}} - \norm{\dot{x}_d}_{max}}{\norm{e}_{max}} \\
\norm{J^\dagger}_{max} = 4.277 \qquad
\norm{\dot{x}_d}_{max} = 0.2356 \qquad
\norm{e}_{max} = 0.1417 \\
K_{base} \approx 3.3
\end{gather*}

Os valores $ \norm{J^\dagger}_{max} $, $ \norm{\dot{x}_d}_{max} $ e $ \norm{e}_{max} $ foram calculados graficamente na simulação, considerando o valor no instante inicial ($ \theta(0) $), onde se mostrou ser o maior valor em todos os casos.

Após esse passo, foi feita a sintonia fina, chegando o mais próximo possível do valor máximo do sinal de controle em cada um dos casos. Os valores encontrados podem ser vistos na tabela \ref{tab:ex1_ganhos}.

\begin{table}[!ht]
\centering
\caption{Ganhos para os controles com e sem Feed-forward em todas as 3 trajetórias.}
\label{tab:ex1_ganhos}

\begin{tabular}{|c|c|c|}
\hline
Controles  & FB & FF \\ \hline
Trajetória 1     & $8.91$ & $7.24$ \\ \hline
Trajetória 2     & $8.91$ & $8.57$ \\ \hline
Trajetória 3     & $8.91$ & $8.57$ \\ \hline
\end{tabular}
\end{table}


\subsection{Resultados}

Cada uma das trajetórias tem dois conjuntos de figuras representando os dois controles pedidos (com e sem feed-forward). 

Desses conjuntos, a figura (a) exibe a trajetórias desejada (em vermelho) e a trajetória simulada (em azul). A figura (b), por sua vez, exibe o sinal do erro ($x_d - x$). Já a figura (c) representa o sinal de controle $u$ para cada uma das 4 juntas controladas.

Como esperado, o controle sem feed-forward não é capaz de tirar o erro em regime de estado estacionário, mesmo sendo muito pequeno. Os sinais de controle atendem à restrição pedida, sempre se aproximando o máximo possível do valor máximo.

Pode-se observar na tabela \ref{tab:ex1_ganhos} que os ganhos para o controle sem ação feed-forward são iguais para todas as trajetórias, sendo limitado apenas pela restrição do sinal de controle. Porém, mesmo aumentando o valor do ganho não seria possível eliminar o erro em EE.

No caso de não ser possível calcular a derivada da trajetória e, dependendo do caso de aplicação do manipulador, o controlador apenas com feed-back de posição pode ser suficiente, já que o erro em EE é pequeno comparado ao tamanho dos elos.


\begin{figure}[!ht]
\centering
  \begin{minipage}{\linewidth}
  \centering
    \begin{subfigure}[b]{0.4\textwidth}
    		\includegraphics[width=1\textwidth]{figs/ex1_a_2_x.pdf}
    		\caption{$x_d$ e $x$}
    		\label{fig:ex1_a_2_x}
    \end{subfigure}
  \end{minipage}
  \begin{minipage}{\linewidth}
  \centering
    \begin{subfigure}[b]{0.4\textwidth}
    		\includegraphics[width=1\textwidth]{figs/ex1_a_2_e.pdf}
    		\caption{$e = x_d - x$}
    		\label{fig:ex1_a_2_e}
    \end{subfigure}
    \begin{subfigure}[b]{0.4\textwidth}
    \includegraphics[width=1\textwidth]{figs/ex1_a_2_dq.pdf}
    \caption{$u = \dot{\theta}$}
    \label{fig:ex1_a_2_dq}
    \end{subfigure}
  \end{minipage}
\caption{Ex 1: Trajetória $a$, controle sem FF.}
\label{fig:ex1_a_2}
\end{figure}

\begin{figure}[!ht]
\centering
  \begin{minipage}{\linewidth}
  \centering
    \begin{subfigure}[b]{0.4\textwidth}
    \includegraphics[width=1\textwidth]{figs/ex1_a_1_x.pdf}
    \caption{$x_d$ e $x$}
    \label{fig:ex1_a_1_x}
    \end{subfigure}
  \end{minipage}
  \begin{minipage}{\linewidth}
  \centering
    \begin{subfigure}[b]{0.4\textwidth}
    \includegraphics[width=1\textwidth]{figs/ex1_a_1_e.pdf}
    \caption{$e = x_d - x$}
    \label{fig:ex1_a_1_e}
    \end{subfigure}
    \begin{subfigure}[b]{0.4\textwidth}
    \includegraphics[width=1\textwidth]{figs/ex1_a_1_dq.pdf}
    \caption{$u = \dot{\theta}$}
    \label{fig:ex1_a_1_dq}
    \end{subfigure}
  \end{minipage}
\caption{Ex 1: Trajetória $a$, controle com FF.}
\label{fig:ex1_a_1}
\end{figure}


\begin{figure}[!ht]
\centering
  \begin{minipage}{\linewidth}
  \centering
    \begin{subfigure}[b]{0.4\textwidth}
    \includegraphics[width=1\textwidth]{figs/ex1_b_2_x.pdf}
    \caption{$x_d$ e $x$}
    \label{fig:ex1_b_2_x}
    \end{subfigure}
  \end{minipage}
  \begin{minipage}{\linewidth}
  \centering
    \begin{subfigure}[b]{0.4\textwidth}
    \includegraphics[width=1\textwidth]{figs/ex1_b_2_e.pdf}
    \caption{$e = x_d - x$}
    \label{fig:ex1_b_2_e}
    \end{subfigure}
    \begin{subfigure}[b]{0.4\textwidth}
    \includegraphics[width=1\textwidth]{figs/ex1_b_2_dq.pdf}
    \caption{$u = \dot{\theta}$}
    \label{fig:ex1_b_2_dq}
    \end{subfigure}
  \end{minipage}
\caption{Ex 1: Trajetória $b$, controle sem FF.}
\label{fig:ex1_b_2}
\end{figure}


\begin{figure}[!ht]
\centering
  \begin{minipage}{\linewidth}
  \centering
    \begin{subfigure}[b]{0.4\textwidth}
    \includegraphics[width=1\textwidth]{figs/ex1_b_1_x.pdf}
    \caption{$x_d$ e $x$}
    \label{fig:ex1_b_1_x}
    \end{subfigure}
  \end{minipage}
  \begin{minipage}{\linewidth}
  \centering
    \begin{subfigure}[b]{0.4\textwidth}
    \includegraphics[width=1\textwidth]{figs/ex1_b_1_e.pdf}
    \caption{$e = x_d - x$}
    \label{fig:ex1_b_1_e}
    \end{subfigure}
    \begin{subfigure}[b]{0.4\textwidth}
    \includegraphics[width=1\textwidth]{figs/ex1_b_1_dq.pdf}
    \caption{$u = \dot{\theta}$}
    \label{fig:ex1_b_1_dq}
    \end{subfigure}
  \end{minipage}
\caption{Ex 1: Trajetória $b$, controle com FF.}
\label{fig:ex1_b_1}
\end{figure}


\begin{figure}[!ht]
\centering
  \begin{minipage}{\linewidth}
  \centering
    \begin{subfigure}[b]{0.4\textwidth}
    \includegraphics[width=1\textwidth]{figs/ex1_c_2_x.png}
    \caption{$x_d$ e $x$}
    \label{fig:ex1_c_2_x}
    \end{subfigure}
  \end{minipage}
  \begin{minipage}{\linewidth}
  \centering
    \begin{subfigure}[b]{0.4\textwidth}
    \includegraphics[width=1\textwidth]{figs/ex1_c_2_e.pdf}
    \caption{$e = x_d - x$}
    \label{fig:ex1_c_2_e}
    \end{subfigure}
    \begin{subfigure}[b]{0.4\textwidth}
    \includegraphics[width=1\textwidth]{figs/ex1_c_2_dq.pdf}
    \caption{$u = \dot{\theta}$}
    \label{fig:ex1_c_2_dq}
    \end{subfigure}
  \end{minipage}
\caption{Ex 1: Trajetória $c$, controle sem FF.}
\label{fig:ex1_c_2}
\end{figure}


\begin{figure}[!ht]
\centering
  \begin{minipage}{\linewidth}
  \centering
    \begin{subfigure}[b]{0.4\textwidth}
    \includegraphics[width=1\textwidth]{figs/ex1_c_1_x.png}
    \caption{$x_d$ e $x$}
    \label{fig:ex1_c_1_x}
    \end{subfigure}
  \end{minipage}
  \begin{minipage}{\linewidth}
  \centering
    \begin{subfigure}[b]{0.4\textwidth}
    \includegraphics[width=1\textwidth]{figs/ex1_c_1_e.pdf}
    \caption{$e = x_d - x$}
    \label{fig:ex1_c_1_e}
    \end{subfigure}
    \begin{subfigure}[b]{0.4\textwidth}
    \includegraphics[width=1\textwidth]{figs/ex1_c_1_dq.pdf}
    \caption{$u = \dot{\theta}$}
    \label{fig:ex1_c_1_dq}
    \end{subfigure}
  \end{minipage}
\caption{Ex 1: Trajetória $c$, controle com FF.}
\label{fig:ex1_c_1}
\end{figure}


\section{Controle Cinemático de Manipuladores Redundantes} % 1

Diferente do manipulador anterior, o objetivo agora é controlar um manipulador redundante. Onde o Jacobiano perde posto por ter mais juntas do que graus de liberdade para a tarefa. 

Nesse caso aparece um comportamento muito interessante, o espaço nulo do Jacobiano, que permite otimizar um grau de liberdade adicional.

A função objetivo a ser otimizada pode ser de diversos tipos. Nesse trabalho, as escolhidas serão a \textbf{orientação},  a \textbf{manipulabilidade} e o \textbf{limite das juntas}.

O manipulador proposto para estudar esse comportamento é o manipulador planar 3R, com comprimento dos 3 elos de $0.5m$.

Assim como foi feito para o manipulador anterior, é necessário utilizar o método de Denavit-Hartenberg, cujos parâmetros estão na tabela \ref{tab:ex2_dh}.

A posição inicial escolhida foi $ q_0 = [\pi, -\pi/2, -\pi/2]^T $, representada na figura \ref{fig:ex2_ready}.

Neste caso, a trajetória desejada é um círculo de raio $0.25m$ com centro no ponto $[0.25 0.5]$. Parametrizada por $x_d$.


   	\begin{gather*}
	x_{d} = 
	\begin{bmatrix} 
	0.25 (1 - cos(\pi t)) \\ 
	0.25 (2 + sin(\pi t)) \\
	\end{bmatrix} \qquad t \in [0,4] \\
	\end{gather*}


\begin{table}[!ht]
\centering
\caption{Tabela de Denavit-Hartenberg do Manipulador Planar 3R.}
\label{tab:ex2_dh}

\begin{tabular}{|c|c|c|c|c|c|}
\hline
Junta  & $\alpha (rad)$ & $A (m)$ & $\theta (rad)$ & $D (m)$ & $Offset (rad)$ \\ \hline
$1$  & 0 & $0.5$ & $\theta_1$ & 0 & 0 \\ \hline
$2$  & 0 & $0.5$ & $\theta_2$ & 0 & 0 \\ \hline
$3$  & 0 & $0.5$ & $\theta_3$ & 0 & 0 \\ \hline

\end{tabular}
\end{table}

\begin{figure}[!ht]
\centering
\includegraphics[width=0.5\textwidth]{figs/ex2_planar_tetha0.pdf}
\caption{Manipulador Planar 3R na posição inicial}
\label{fig:ex2_ready}
\end{figure}

\subsection{Controle}

Este caso é caracterizado por o numero de juntas $n$ ser maior do que o número de graus de liberdade da tarefa $m$. O que traria a necessidade de usar novamente a pseudo-inversa.

Porém, diferente do caso anterior, esse tem outra particularidade. Como o manipulador é planar com todas as juntas de rotação no eixo $z$, duas linhas do Jacobiano serão nulas. Além disso, por todas as juntas estarem em todo o tempo no plano XY, a terceira linha do Jacobiano, referente a posição no eixo $z$, também será zero.

Com isso, pode-se trabalhar com um Jacobiano reduzido chamado de Jacobiano Analítico ($J_a \in R^{3x3}$). Por $J_a$ ser quadrado e não singular, ele é inversível.

Ao reduzir o Jacobiano, também se faz necessário reduzir o sinal de controle ($u \in R^{6}$), passando a ter dimensão 3.


\subsubsection{Orientação}

No primeiro caso de controle, pede-se para que a orientação também siga uma trajetória desejada ($\phi_d$).

Como foi adicionado mais um grau de liberdade, o sistema passa a ter 3 graus de liberdade (posição x, y e orientação) e tem 3 graus de mobilidade, formado por suas 3 juntas. Trabalhando com o Jacobiano Analítico, passamos a ter um Jacobiano quadrado não-singular, logo, inversível.

\begin{gather*}
	u = \dot{\theta} \qquad J_a \in R^{3x3} \\
	\dot{x} = J_a(\theta)\dot{\theta} = J_a(\theta)u \quad \Rightarrow \quad
	u = J_a(\theta)^{-1}\dot{x} \\
	u = J_a(\theta)^{-1} \bar{u} \\
	\bar{u} = \dot{x_d} + K (x_d - x) \\
	x_d = 
	\begin{bmatrix} 
	p_d \\ 
	\phi_d \\
	\end{bmatrix} \qquad 
	\phi_d = \theta_1 + \theta_2 + \theta_3 = sin(\pi t / 24) \\
	p_{d} = 
	\begin{bmatrix} 
	0.25 (1 - cos(\pi t)) \\ 
	0.25 (2 + sin(\pi t)) \\
	0 \\
	\end{bmatrix}, \qquad 
	\phi_{d} = 
	\begin{bmatrix} 
	0 \\ 
	0 \\
	sin(\pi t / 24) \\
	\end{bmatrix} \quad \Rightarrow \quad 
	\dot{\phi_d} = 
	\begin{bmatrix} 
	0 \\
	0 \\
	\dfrac{\pi cos(\pi t / 24)}{24} \\
	\end{bmatrix} \qquad t \in [0,4] \\
	x_d = 
	\begin{bmatrix} 
	x \\ 
	y \\
	z \\
	\phi_1 \\
	\phi_2 \\
	\phi_3 \\
	\end{bmatrix}, \quad
	x_d = 
	\begin{bmatrix} 
	x \\ 
	y \\
	0 \\
	0 \\
	0 \\
	\phi_3 \\
	\end{bmatrix}
\end{gather*}
 
\subsubsection{Funções Objetivo}

Diferente do caso anterior, os próximos casos não tem a orientação para ser controlada, logo, voltaria a ser um sistema com apenas dois graus de liberdade, as posições em x e em y. Porém, a ideia agora é adicionar mais um grau de liberdade, utilizando o \textbf{espaço nulo do Jacobiano}.

Para se controlar o sistema com 3 graus de liberdade, adiciona-se uma nova parcela no sinal de controle, o espaço nulo do Jacobiano, que é alcançado usando-se a pseudo-inversa dele. A variável $\mu$, o grau de liberdade adicional, é calculada multiplicando-se um ganho $K_0$ pela derivada da função $\omega(\theta)$ a ser otimizada.

Isso não altera a equação do erro.

\begin{gather*}
	\bar{u} = J^{\dagger}[\dot{x_d} + K(x_d - x)] + (I - J^{\dagger}J)\mu \\
	\dot{e} + Ke = 0 \\
	\mu = K_0\left( \dfrac{\partial \omega(\theta)}{\partial \theta} \right)^T  \\
\end{gather*}

\quad \textbullet \quad Manipulabilidade

A primeira função objetivo é a controlabilidade, que consiste numa medida de distância que o manipulador está de uma posição de singularidade do Jacobiano.

\begin{gather*}
	\omega(\theta) = 
	\begin{bmatrix} 
	0 \\
	0.5 sin^2(\theta_2) \\
	0.5 sin^2(\theta_3) \\
	\end{bmatrix} \quad \Rightarrow \quad 
	\dot{\omega}(\theta) = 
	\begin{bmatrix} 
	0 \\
	sin(\theta_2)cos(\theta_2) \\
	sin(\theta_3)cos(\theta_3) \\
	\end{bmatrix} \quad \Rightarrow \quad 
	\mu = K_0\begin{bmatrix} 
	0 \\
	sin(\theta_2)cos(\theta_2) \\
	sin(\theta_3)cos(\theta_3) \\
	\end{bmatrix}\\
\end{gather*}

\quad \textbullet \quad Limite das Juntas

O segundo caso de função objetivo é o limite de cada uma das juntas.

\begin{gather*}
	-2 \pi \leq \theta_1 \leq 2\pi; \quad -\pi/2 \leq \theta_2 \leq \pi/2; \quad -3\pi/2 \leq \theta_3 \leq -\pi/2 \\
	\omega(\theta) = - \dfrac{1}{2n}\sum_{i=1}^{n} \left( \dfrac{\theta_i - \bar{\theta}_i}{\theta_{iM} - \theta_{im}} \right)^2 \\
	\omega(\theta) = 
	\begin{bmatrix} 
	- \frac{\theta_1^2}{96\pi^2} \\
	- \frac{\theta_2^2}{6\pi^2} \\
	- \frac{(\pi + \theta_3)^2}{6\pi^2} \\
	\end{bmatrix} \quad \Rightarrow \quad 
	\dot{\omega}(\theta) = 
	\begin{bmatrix} 
	- \frac{\theta_1}{48\pi^2} \\
	- \frac{\theta_2}{3\pi^2} \\
	- \frac{\pi + \theta_3}{3\pi^2} \\
	\end{bmatrix} \quad \Rightarrow \quad 
	\mu = K_0\begin{bmatrix} 
	- \frac{\theta_1}{48\pi^2} \\
	- \frac{\theta_2}{3\pi^2} \\
	- \frac{\pi + \theta_3}{3\pi^2} \\
	\end{bmatrix}\\
\end{gather*}

\newpage

\subsection{Simulações}


\begin{figure}[!ht]
\centering
\includegraphics[width=0.9\textwidth]{figs/ex2_1_simulink}
\caption{Topologia do Simulink Controle de Orientação.}
\label{fig:ex2_1_simulink}
\end{figure}

\begin{figure}[!ht]
\centering
\includegraphics[width=0.9\textwidth]{figs/ex2_2a_simulink}
\caption{Topologia do Simulink Manipulabilidade.}
\label{fig:ex2_2a_simulink}
\end{figure}

\begin{figure}[!ht]
\centering
\includegraphics[width=0.9\textwidth]{figs/ex2_2b_simulink}
\caption{Topologia do Simulink Limite das Juntas.}
\label{fig:ex2_2b_simulink}
\end{figure}

\subsection{Resultados}

Nos três casos, qualquer ganho positivo seria capaz de levar o erro para zero. E, como se pode ver, os erros são praticamente zero.

Os manipuladores estão sendo exibidos na posição final, para que se pudesse ter uma ideia visual da trajetória percorrida.

Como o problema não apresenta restrições, foi escolhido um ganho relativamente alto. Isso é importante para o Limite das Juntas, que, para um ganho baixo, não é capaz de atender à especificação.

Mesmo com o erro sendo praticamente zero em todos os casos, as juntas se comportam de maneira muito distintas. Esse comportamento pode ser visto no gráfico (b) de cada controle.

Um comportamento importante, que não é fácil de analisar apenas pelos gráficos, é a colisão dos elos. No controle de orientação restrita, os elos chegam muito perto de colidir. E, na otimização do Limites das Juntas, os elos chegam a se colidir.

Pode-se comparar a \textbf{manipulabilidade} entre o caso da controle orientação e a otimização da manipulabilidade. No primeiro caso, a manipulabilidade chega bem próximo de zero duas vezes, nesses instantes, pode-se verificar picos nos gráficos do sinal de controle e no erro. Já no caso da otimização, a manipulabilidade tem um comportamento muito mais confortável, tendo seu menor valor maior do que $0.9$.

% Controle 1
\begin{figure}[!ht]
\centering
  \begin{minipage}{\linewidth}
  \centering
    \begin{subfigure}[b]{1\textwidth}
    \centering
    \includegraphics[width=0.6\textwidth]{figs/ex2_1_traj.pdf}
    \caption{Trajetória}
    \label{fig:ex2_1_traj}
    \end{subfigure}
  \end{minipage}
  \begin{minipage}{\linewidth}
  \centering
    \begin{subfigure}[b]{0.49\textwidth}
    \includegraphics[width=1\textwidth]{figs/ex2_1_q.pdf}
    \caption{Ângulo das juntas}
    \label{fig:ex2_1_q}
    \end{subfigure}
    \begin{subfigure}[b]{0.49\textwidth}
    \includegraphics[width=1\textwidth]{figs/ex2_1_e.pdf}
    \caption{Erro da posição do efetuador}
    \label{fig:ex2_1_e}
    \end{subfigure}
  \end{minipage}
  \begin{minipage}{\linewidth}
  \centering
    \begin{subfigure}[b]{0.49\textwidth}
    \includegraphics[width=1\textwidth]{figs/ex2_1_dq.pdf}
    \caption{Sinal de controle/velocidade das juntas}
    \label{fig:ex2_1_dq}
    \end{subfigure}
    \begin{subfigure}[b]{0.49\textwidth}
    \includegraphics[width=1\textwidth]{figs/ex2_1_manip.pdf}
    \caption{Manipulabilidade}
    \label{fig:ex2_1_manip}
    \end{subfigure}
  \end{minipage}
\caption{Ex 2: Orientação, K = 50}
\label{fig:ex2_1}
\end{figure}

% Controle 1
\begin{figure}[!ht]
\centering
  \begin{minipage}{\linewidth}
  \centering
    \begin{subfigure}[b]{1\textwidth}
    \centering
    \includegraphics[width=0.6\textwidth]{figs/ex2_2a_traj.pdf}
    \caption{Trajetória}
    \label{fig:ex2_2a_traj}
    \end{subfigure}
  \end{minipage}
  \begin{minipage}{\linewidth}
  \centering
    \begin{subfigure}[b]{0.49\textwidth}
    \includegraphics[width=1\textwidth]{figs/ex2_2a_q.pdf}
    \caption{Ângulo das juntas}
    \label{fig:ex2_2a_q}
    \end{subfigure}
    \begin{subfigure}[b]{0.49\textwidth}
    \includegraphics[width=1\textwidth]{figs/ex2_2a_e.pdf}
    \caption{Erro da posição do efetuador}
    \label{fig:ex2_2a_e}
    \end{subfigure}
  \end{minipage}
  \begin{minipage}{\linewidth}
  \centering
    \begin{subfigure}[b]{0.49\textwidth}
    \includegraphics[width=1\textwidth]{figs/ex2_2a_dq.pdf}
    \caption{Sinal de controle/velocidade das juntas}
    \label{fig:ex2_2a_dq}
    \end{subfigure}
    \begin{subfigure}[b]{0.49\textwidth}
    \includegraphics[width=1\textwidth]{figs/ex2_2a_manip.pdf}
    \caption{Manipulabilidade}
    \label{fig:ex2_2a_manip}
    \end{subfigure}
  \end{minipage}
\caption{Ex 2: Manipulabilidade, K = 50, $K_0$ = 50}
\label{fig:ex2_2a}
\end{figure}

% Controle 1
\begin{figure}[!ht]
\centering
  \begin{minipage}{\linewidth}
  \centering
    \begin{subfigure}[b]{1\textwidth}
    \centering
    \includegraphics[width=0.6\textwidth]{figs/ex2_2b_traj.pdf}
    \caption{Trajetória}
    \label{fig:ex2_2b_traj}
    \end{subfigure}
  \end{minipage}
  \begin{minipage}{\linewidth}
  \centering
    \begin{subfigure}[b]{0.49\textwidth}
    \includegraphics[width=1\textwidth]{figs/ex2_2b_q.pdf}
    \caption{Ângulo das juntas}
    \label{fig:ex2_2b_q}
    \end{subfigure}
    \begin{subfigure}[b]{0.49\textwidth}
    \includegraphics[width=1\textwidth]{figs/ex2_2b_e.pdf}
    \caption{Erro da posição do efetuador}
    \label{fig:ex2_2b_e}
    \end{subfigure}
  \end{minipage}
  \begin{minipage}{\linewidth}
  \centering
    \begin{subfigure}[b]{0.49\textwidth}
    \includegraphics[width=1\textwidth]{figs/ex2_2b_dq.pdf}
    \caption{Sinal de controle/velocidade das juntas}
    \label{fig:ex2_2b_dq}
    \end{subfigure}
    \begin{subfigure}[b]{0.49\textwidth}
    \includegraphics[width=1\textwidth]{figs/ex2_2b_lim_jun.pdf}
    \caption{Limite das Juntas}
    \label{fig:ex2_2b_manip}
    \end{subfigure}
  \end{minipage}
\caption{Ex 2: Limite das Juntas, K = 50, $K_0$ = 50}
\label{fig:ex2_2b}
\end{figure}

\end{document}
